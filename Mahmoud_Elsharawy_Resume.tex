\documentclass{article}
\usepackage{calc}
\usepackage[dvipsnames]{xcolor}
\usepackage[margin=0.5in]{geometry}
\usepackage{hyperref}
\usepackage{enumitem}
\usepackage{fontenc}
\usepackage{ifthen}

\setlist[enumerate]{topsep=0pt}
\setlist[itemize]{noitemsep,topsep=0pt}
\definecolor{header_color}{rgb}{0.22,0.45,0.70}% light blue
\newlength{\remaining}
\renewcommand{\section}[1]{
  \vspace{1em}\setlength{\remaining}{\textwidth-\widthof{\fontsize{14}{15}\bfseries\uppercase{#1}}}\noindent\textcolor{header_color}{\underline{\fontsize{14}{20}\bfseries\uppercase{#1}\hspace*{\remaining}}}\\
}

\renewcommand{\subsection}[1]{
  % \vspace{0.4em}
  \noindent{\textcolor{header_color}{\fontsize{12}{15}\bfseries{#1}\dotfill}}\\
}

\renewcommand{\subsubsection}[3]{
    \noindent\fontsize{12}{15}{\selectfont \textbf{#1} \ifthenelse{\equal{#2}{}}{}{| \emph{#2}} \hfill #3}
}

\newcommand{\eduitem}[3]{
    \vspace{0.3cm}\\\noindent\fontsize{12}{15}{\selectfont\textbf{#1} \emph{#2} \hfill #3\\}
}
\pagenumbering{gobble}

\begin{document}
    \fontencoding{T1}
    \fontfamily{lmr} %phv
    \fontsize{11}{13} %9 15
    \selectfont
    \begin{center}
        \begin{center}
            \Huge\bfseries \textcolor{header_color}{Mahmoud Elsharawy}
        \end{center}
%            \begin{tabular}{c c c c c}
        \begin{tabular}{c c c c c}
                \href{https://mse63.github.io/}{https://mse63.github.io/} &
                $\bullet$ &
                \href{tel:929-461-9837}{(929)-461-9837}&
                $\bullet$ &
                \href{mailto:mse63@cornell.edu}{mse63@cornell.edu} \\
        \end{tabular}
    \end{center}
    \vspace{-0.75em}
    %----------------------------------------------------------------------------------------
    %	EDUCATION SECTION
    %----------------------------------------------------------------------------------------
    \section{Education}
    \vspace{-0.6cm}
    \eduitem{Cornell University}{}{Expected 05/2025}
    Master of Engineering in Computer Science \hfill
    \eduitem{Cornell University}{}{Expected 12/2024}
    B.S. in Computer Science, B.S. in Electrical and Computer Engineering \hfill GPA: 3.80\\
    \subsection{Relevant Coursework}
    \begin{tabular}{l l l}
       Operating Systems  & Reinforcement Learning & Digital System Design Using Microcontrollers \\
       Computer Architecture & Machine Learning & Analysis of Algorithms \\
    \end{tabular}\\

    \subsection{Skills}
    \textbf{Programming Languages: } Rust, Java, Bash, Python, C/C++\\
    \textbf{Software: } Linux, LTspice, Cadence, KiCAD, Altium, Mentor Designer/Layout, Quartus\\
    \textbf{Hardware: } Raspberry Pi, Arduino, FPGA Boards, Verilog

    %----------------------------------------------------------------------------------------
    %	EXPERIENCE SECTION
    %----------------------------------------------------------------------------------------
    \section{Experience}
    \subsubsection{Apple}{System Electrical Engineering Intern}{Jan 2023 - Aug 2023}
    \begin{itemize}
        \item Designed and tested a buck converter power module for use on internal dev boards, removing reliance on a vendor's power modules, preventing future supply chain issues and reducing cost
        \item Designed a PCBA to calibrate the ADC of a SAMD21 microcontroller, and programmed it using C to act as a micro-current load to precisely characterize power components
        \item Created prototype analog and RF circuits for future development
        \item Coordinated with a Product Design Engineer, DFM, and PCB Designer to design flexible PCBAs for prototypes
    \end{itemize}

    \subsubsection{SpaceX}{Hardware Engineering Intern}{Jan 2022 - Aug 2022}
    \begin{itemize}
        \item Anchored User Terminal Power Budget over temperature and operating mode by automating thermal chamber data collection through SCPI commands, informing thermal team of shortcomings and improving field predictions
        \item Automated an assembly line station through mechanical design and PLC TwinCAT software, tripling its speed and preventing a production bottleneck
        \item Tested and qualified alternative integrated circuits for Business User Terminals, preventing a parts shortage
        \item Designed a dev PCBA to test various potential fixes for acoustic noise from user terminals
    \end{itemize}

    \subsubsection{Cornell University Unmanned Air Systems}{Electrical Team Member}{Aug 2022 - Dec 2022}
    \begin{itemize}
        \item Designed a PCB using an ESP32-S2 microcontroller programmed in C to interface with a camera over a DVP or USB protocol and save footage to an SD card
    \end{itemize}

    %----------------------------------------------------------------------------------------
    %	PROJECTS SECTION
    %----------------------------------------------------------------------------------------
    \section{Projects}
    \subsubsection{Chess AI}{}{Aug 2021 - Jun 2022}
    \begin{itemize}
        \item Developed a UCI Chess AI from scratch in Rust using a minimax algorithm with variable depth and time control
        \item Hosted a systemd service to interact with lichess.com's API, allowing the AI to play against other bots and humans, earning an Elo rating of 1700, making it stronger than 70\% of human players on the website
    \end{itemize}

    \subsubsection{Servo Controller}{}{Aug 2021 - Jun 2022}
    \begin{itemize}
        \item Modified servos to provide an analog feedback signal of their position by accessing the potentiometer within it
        \item Designed and created an Op-Amp circuit which generates a pulse width modulation (PWM) signal, controlling the position of a servo to match that of another servo by implementing an analog feedback loop to adjust the signal
    \end{itemize}

    \subsubsection{Automatic Plant Waterer}{}{Aug 2021 - Jun 2022}
    \begin{itemize}
        \item Designed, programmed and built an Arduino-controlled 3D-Printed automatic plant watering machine with C++, housing a mint plant and a water supply
        \item Implemented an Arduino to detect when the soil is too dry and use a peristaltic pump to water the plant when necessary
    \end{itemize}
\end{document}
